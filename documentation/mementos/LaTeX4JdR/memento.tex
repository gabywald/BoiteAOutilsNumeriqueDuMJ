\documentclass[11pt,twoside,a4paper]{article}
% http://www-h.eng.cam.ac.uk/help/tpl/textprocessing/latex_maths+pix/node6.html symboles de math
% http://fr.wikibooks.org/wiki/Programmation_LaTeX Programmation latex (wikibook)
%=========================== En-Tete =================================
%--- Insertion de paquetages (optionnel) ---
%--- Insertion de paquetages (optionnel) ---
\usepackage[french]{babel}   % pour dire que le texte est en fran{\'e}ais
\usepackage{a4}	             % pour la taille   
\usepackage[T1]{fontenc}     % pour les font postscript
\usepackage{epsfig}          % pour gerer les images
%\usepackage{psfig}
\usepackage{amsmath, amsthm} % tres bon mode mathematique
\usepackage{amsfonts,amssymb}% permet la definition des ensembles
\usepackage{float}           % pour le placement des figure
\usepackage{verbatim}

\usepackage{longtable} % pour les tableaux de plusieurs pages

\usepackage[table]{xcolor} % couleur de fond des cellules de tableaux

\usepackage{lastpage}

\usepackage{multirow}

\usepackage{multicol} % pour {\'e}crire dans certaines zones en colonnes : \begin{multicols}{nb colonnes}...\end{multicols}

%% https://texblog.org/2011/02/26/generating-dummy-textblindtext-with-latex-for-testing/
%% https://fr.sharelatex.com/learn/Multiple_columns
%% \usepackage{blindtext}
%% \usepackage{lipsum}
\usepackage{wrapfig}

% \usepackage[top=1.5cm, bottom=1.5cm, left=1.5cm, right=1.5cm]{geometry}
% gauche, haut, droite, bas, entete, ente2txt, pied, txt2pied
\usepackage{vmargin}
\setmarginsrb{1.0cm}{1.0cm}{1.0cm}{1.0cm}{15pt}{3pt}{57pt}{3pt}

\usepackage{lscape} % changement orientation page
\usepackage{pdflscape}
% --- style de page (pour les en-tete) ---
\pagestyle{headings}
% \pagestyle{empty}

% % % en-tete et pieds de page configurables : fancyhdr.sty

% http://www.trustonme.net/didactels/250.html

% http://ww3.ac-poitiers.fr/math/tex/pratique/entete/entete.htm
% http://www.ctan.org/tex-archive/macros/latex/contrib/fancyhdr/fancyhdr.pdf
% \usepackage{fancyhdr}
% \pagestyle{fancy}
% % \newcommand{\chaptermark}[1]{\markboth{#1}{}}
% % \newcommand{\sectionmark}[1]{\markright{\thesection\ #1}}
% \fancyhf{}
% \fancyhead[LE,RO]{\bfseries\thepage}
% \fancyhead[LO]{\bfseries\rightmark}
% \fancyhead[RE]{\bfseries\leftmark}
% \fancyfoot[LE]{\thepage /\pageref{LastPage} \hfill
	% TITLE
% \hfill \includegraphics[width=0.5cm]{img/logo_glider.png} }
% \fancyfoot[RO]{\includegraphics[width=0.5cm]{img/logo_glider.png} \hfill
	% TITLE
% \hfill \thepage /\pageref{LastPage}}
% \renewcommand{\headrulewidth}{0.5pt}
% \renewcommand{\footrulewidth}{0.5pt}
% \addtolength{\headheight}{0.5pt}
% \fancypagestyle{plain}{
	% \fancyhead{}
	% \renewcommand{\headrulewidth}{0pt}
% }

%--- Definitions de nouvelles commandes ---
\newcommand{\N}{\mathbb{N}} % les entiers naturels

%============================= Corps =================================
\begin{document}
\begin{landscape}

% \setcounter{page}{0}
% \thispagestyle{empty}

\begin{multicols}{2}

	\textsc{\Huge Memento LaTeX pour le JdR}~\\
	
	\tableofcontents
	
	%% \vfill~\\
	%% \columnbreak
	
	%% \section{Idées de base}
	%% \begin{itemize}
		%% \item En-Tête, type de documents (livre, article, lettre...) ; 
		%% \item Comment compiler un document LaTeX ?
		%% \item Mise en page (par défaut, changements possibles, boites...) ;
		%% \item Table des matières : chapter, section, subsection, subsection... ; 
		%% \item Table des figures ; 
		%% \item Table de tableaux ; 
		%% \item Table des index et références ; 
		%% \item Insertions figures / images : ... ; 
		%% \item En-tête et Pied de page : types par défaut et personnalisation (fancyhdr) ; 
		%% \item Tableaux, Tableaux Longs ;  
		%% \item Polices de caractères / Fontes : usages, changements... ;
		%% \item Tikz : décoration, frises, schémas, texte en travers ; 
		%% \item Dessiner des personnages simples avec le package \texttt{tikzpeople}... ;
		%% \item Tirage de cartes avec le package \texttt{JeuxCarte} ; 
		%% \item Beamer (faire des présentations)... ; 
	%% \end{itemize}
	
	\section{Rappel de règles de rédaction}
	
	\textbf{\textsc{Tu veux publier un document ? Tu es au bon endroit !}}~\\
	
	Quelques rappels tout de même, que \LaTeX~ facilitera sans faire à ta place : 
	\begin{itemize}
		\item Accepte les relectures et corrections !
		\item L'orthographe et la grammaire sont importants (et essentiels), surtout pour que tes écrits persistent dans le temps et qu'ils puissent être facilement relus par d'autres (y compris le <<toi du futur>>) ;~\newline
			$\rightarrow$ Fait-toi relire (à défaut d'un correcteur orthographique correct) !!
		\item Concernant le style d'écriture : tant que tu trouves un public pour te lire et apprécier ton contenu, tout ira bien (sinon, il faut en changer). 
	\end{itemize}

	%% \vfill~\\
	%% \columnbreak
	
	\section{Introduction à \LaTeX}
	
	\subsection{Ce que permet \LaTeX}
	
	\begin{itemize}
		\item Offre une configuration par défaut et une mise en page conforme aux normes et standards de lisibilité et d'impression par défaut ; 
		\item Faire un minimum de mise en page sans trop réfléchir au départ (taille des caractères, police de caractère / fonte, justification du texte / égalisation sur les côtés par défaut...
		\item Des éléments de mise en page qui maintiennent la lisibilité tout en étant inclus dans le document. 
		\item Utiliser des documents en texte brut (avec un logiciel du type de NotePad++, TextEditor, VIm, Nano, Emacs, Geany, Gedit, Jedit...) : 
		\begin{itemize}
			\item Utilise moins de place sur le disque dur ; 
			\item Compatible avec n'importe quel éditeur sur n'importe quel système ; 
			\item Éditable en WYSIWIG avec LyX ; 
			\item De la documentation et une communauté disponible en ligne (ou pas) et éprouvée ; 
			\item ...
		\end{itemize}
	\end{itemize}

	\subsection{Ce que NE permet PAS \LaTeX}

	\begin{itemize}
		\item Écrire à ta place. 
		\item Avoir un texte correct pour la grammaire et l'orthographe. 
		\item Avoir un style correct / lisible. 
		\item Avoir des idées à ta place (et empêcher d'en avoir). 
		\item ...
	\end{itemize}
	
	%% \vfill~\\
	%% \columnbreak
	
	\section{Bases : En-Tête, type de documents (livre, lettre...)}
	\section{Comment compiler un document LaTeX ?}
	\section{Mise en page (par défaut, changements possibles, boites...)} 
	\section{Table des matières : part, chapter, section, subsection...} 
	\section{Table des figures} 
	\section{Table de tableaux} 
	\section{Table des index et références} 
	\section{Insertions figures / images}  
	\section{En-tête et Pied de page : types par défaut et personnalisation} 
	\section{Tableaux, Tableaux Longs} 
	\section{Tikz : décoration, frises, schémas, texte en travers...}  
	\section{Polices de caractères / Fontes : usages, changements...} 
	\section{Dessiner des personnages simples avec le package \texttt{tikzpeople}...}
	\section{Tirage de cartes avec le package \texttt{JeuxCarte}...}
	\section{Beamer (faire des présentations)...}
	
	...
\end{multicols}

\clearpage

\end{landscape}
\end{document}
